
\documentclass{beamer}

\newcommand*{\mybox}[1]{\framebox{#1}}

\mode<presentation> {
\usetheme{CambridgeUS}
\usecolortheme{dolphin}
}

\usepackage{graphicx} 
\usepackage{booktabs} 

\title[Gerez, Riaz \& Teoldi]{The structure of inequality and politics of redistribution} 

\author{Lupu and Pontusson (2011)} 
\institute[] 
{
Discussion by \\ Julian Enrique Gerez, Zara Riaz \& Filippo Teoldi \\ \medskip
Columbia University
\medskip
}
\date{October 23, 2018} 

\begin{document}
\begin{frame}
\titlepage 
\end{frame}
\begin{frame}
\frametitle{Table of contents} 
\tableofcontents 
\end{frame}
\begin{frame}


\subsection{Aim of the paper}
\frametitle{Aim of the paper}
\begin{itemize}
\item [1.] Does more \textbf{inequality} lead to more \textbf{redistribution}? \\
\medskip
\item[2.] What is the role of \textbf{middle-income voters} on government's redistribution policies?\\
\medskip
$\rightsquigarrow$ Redistributive policy outcome correspond to the policy preferences of middle-income voters\\
\medskip
$\rightsquigarrow$ The structure of inequality helps explain why the preferences of middle-income voters vary across countries/over time
\end{itemize}
\end{frame}

\begin{frame}
\frametitle{Inequalities and Social affinity hypotesis}
\begin{itemize}
\item[•] Structure of inequality, rather than the level: \\
\medskip
\begin{center}
\textbf{Skew} = $\frac{90th/50th}{50th/10th}$ 
\end{center}
\medskip
\item[•] \textbf{Social affinity hypothesis} [Luttermer’s (2001) and Shayo(2009)]: Middle income voters empathize with the poor (affluent) when they perceive the poor (affluent) as living lives similar to their own
\end{itemize}
\medskip
\begin{center}
\textbf{=}
\end{center}
\medskip
$\uparrow$ \mybox{skew of income distribution} $\longmapsto$ distance between middle and poor is smaller (relative to middle and upper) $\longmapsto$ \mybox{asking for $\uparrow$ redistribution} \\
\end{frame}
\begin{frame}
\subsection{Design declaration} 
\frametitle{Design declaration}\
\begin{itemize}


\item[a)] \textit{declare population} = Describes dimensions and distributions over the variables in the population  $\longrightarrow$ The study concerns\textbf{ country year units} (858 observations).
\item[b)] \textit{declare potential outcomes} = Takes population or sample and adds potential outcomes produced by interventions $\longrightarrow$ Does more \textbf{inequality lead to more redistribution}?  
\item[c)] \textit{declare sampling} = (takes a population and selects a sample)  $\longrightarrow$ \textbf{N = 858}
\item[d)] \textit{declare assignment} = (takes a population or sample and adds treatment assignments)  $\longrightarrow$ \textbf{XXX}
\item[e)]\textit{declare estimand} = (takes potential outcomes and calculates a quantity of interest) $\longrightarrow$ \textbf{OLS with robust standard errors}
\item[f)]\textit{declare estimator} = takes data produced by sampling and assignment and returns estimates) $\longrightarrow$ \textbf{XXX}
\end{itemize}
\end{frame}
\begin{frame}

\subsection{Empirical set-up and results}
\frametitle{Empirical set-up}\
$\longrightarrow$ For 15-18 advance democracies over 1969 to 2005 period, they estimate a time-series and cross-section model:\\
\begin{center}
$R_{i,t}$ = $\alpha$ + $\beta \frac{\Sigma ^{s} _{s=1}P_{i, t-s}}{S}$ + $\gamma R_{i, t-1}$ + $\epsilon_{i,t}$
\end{center}
where,
\begin{itemize}
\begin{scriptsize}
\item[-] $R_{i,t}$ is the level of redistribution defined as (a) $\frac{Gini_{Gross} - Gini_{Disposable}}{Gini_{Gross}}$ or Nonelderly social spending in \% of GDP
\item[-] $S$ is the number of years between each observation of redistribution
\item[-]  $P_{i, t-s}$ is a set of policies and structural factors that cause redistribution to deviate from status quo $\longrightarrow$ Control variables: immigration, skills, voting turnout, electoral system, VTR, labor mkt
\end{scriptsize}
\end{itemize}
\end{frame}

\begin{frame}
\frametitle{Empirical results}

[insert here: DATA AND TABLE]\\
\medskip
$\longrightarrow$ \textbf{redistribution increases} with dispersion of the upper half of the earnings distribution and with compression of the lower half of the earnings distribution

\end{frame}

\begin{frame}
\frametitle{Empirical results}
$\longrightarrow$  What about preferences of middle-income voters?
\medskip
\begin{itemize}
\item[(i)] Correlation (R =.45) btw the \textbf{inequality} and \textbf{support for redistribution} of the middle-income voters  
\item[(ii)] Correlation (R =.43)  btw the \textbf{preference} of middle-income voters and \textbf{redistributive policies} pursued by government
\item[(iii)] Skewed \textbf{earnings inequality} promotes \textbf{left participation} in government ($R^{2}$ = .12)
\end{itemize}
\medskip
\begin{center}
= Preliminary result (see R and $R^{2}$ level)\\
\end{center}
\end{frame}


\begin{frame}
\subsection{Robustness check} 
\frametitle{Robustness check}\
\begin{itemize}
\item[•]
\end{itemize}
\end{frame}
\begin{frame}
\subsection{Conclusion}
\frametitle{Conclusion}
\begin{itemize}
\item[1.] The structure of \textbf{inequality} is statistically and significantly associated with more \textbf{redistribution and social spending}
\medskip
\medskip
\medskip
\item[2.] \textbf{Middle-income voters} are incline to allay with low-income voters and support redistributive policies when the distance between the middle and the poor is small (relative to the distance between the middle and the upper)
\medskip
\medskip
\medskip
\item[3.] \textbf{Left-leaning government} are more likely to redistribute income than right-leaning government and that governments are more likely to be left-leaning when the structure of inequality is skewed
\end{itemize}
\end{frame}

\begin{frame}
\subsection{Extensions} 
\frametitle{Extensions}\
\begin{itemize}
\item[•]
\end{itemize}
\end{frame}

\begin{frame}
\subsection{Old slides} 
\frametitle{Old slides}\
\begin{itemize}
\item[•]
\end{itemize}
\end{frame}

\begin{frame}
\frametitle{References}
\footnotesize{
\begin{thebibliography}{99} % Beamer does not support BibTeX so references must be inserted manually as below
\bibitem[Lupu and Pontusson, 2011]{p1} Lupu and Pontusson (2011)
\newblock The structure of Inequalities and the Politics of Redistribution
\newblock \emph{American Political Science Review} 105(2), 316 -- 335.
\bibitem[Alesina and Gleser, 2004]{p1} Alesina and Gleser (2004)
\newblock Fighting Poverty in the US and Europe: A World of Difference
\newblock \emph{Oxford UK: Oxford University Press.} 
\bibitem[Alesina and Perotti, 1994]{p1} Alesina and Perotti, (1994)
\newblock The Political Economy of Growth: A Critical Survey of the Recent Literature
\newblock \emph{World Bank Economic Review 1994, vol. 8, issue 3, 351-71} 


\end{thebibliography}
}
\end{frame}

\end{document}